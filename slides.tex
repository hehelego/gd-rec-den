\documentclass{beamer}
\usetheme{Berlin}
\usecolortheme{seahorse}

\usepackage{amsmath, amssymb, amsthm}
\usepackage{enumerate}

\title{guarded recursion-based denotational sematics}
\subtitle{featuring misleading analogs and suspicious agda code}
\author{Charles Peng\and Yvan Dong}
\date{\today}
\institute{Institute: N/A}

\begin{document}

\maketitle

\section{overview}
\begin{frame}{overview}
	\begin{itemize}
		\item denotational semantics
			\begin{itemize}
				\item what is this all about: a quick demonstration of "denotational semantics of propositional logic". (examples ...)
				\item set-theoretical model of STLC.
				\item the general approach and difficulties of scaling to full-fledged PL.
			\end{itemize}
		\item guarded type theory:
			\begin{itemize}
				\item
					\(F(X) = 1 + A * X\)
					\(F(X) = A * X\)
				\item coinductive fixed point examples:
					intutiively make sense, have corresponding infinite structures e.g. stream/deterministic automata run/NFA omega run ...
					we can come up with the proof rules for those fixed point.
					2 DFA runs => synchronous parallel execution, intersection. (think twice ...)
					% L A [iso] A + L A
				\item guarded recursion: generalize to fixed point of whatever functor.
					% gfix f = f (next (gfix f))
					% later modality, next ...

					% lob : (later A -> A) -> A
					% next : A -> later A
					% Stream A = gfix ((X : later Type) -> A * X)
					% A : Type
					% X : later Type
				\item
					time-elpase intuition:

					type -- a collection of well-structures terms
					(nat -- zero, succ of nat).

					guarded type theory: type  -- a collection of terms that appears to be well-structured upto $k$ unfolds.
					% intuition: the topos of trees model and step-indexed proof interpretation
				\item examples and important constructions (stream).
			\end{itemize}

		% ~ 35 mins
		\item denotational semantics of PCF in guarded type theory.
			\begin{itemize}
				\item why cubical: higher inductive types (finite powerset and its equational reasoning), canonicity of the metatheory, difficulties in connection with the fixed point unfolding equation.

				\item use denotational semantics to decide program equivalence...
					.....
				\item recall framework: do not touch any technical details.


				\item
				\item Lifting monad
					1. [[PCF-Type]] denotation of types.
					[[nat]]
					[[sigma -> tau]] = [[sigma]] -> [[tau]]
					2. [[context |- program : tau]] denotation, [[context]] -> [[tau]]
				\item PCF non-terminating.
					program : nat

					[[program]] : Nat
					[[program]] : bottom + Nat (type theory can decide program termination)
					[[program]] : Lift (Nat)

				\item the key definition and lemmas

					(explain the definition, theorem statement.
					explain proof of one or two intersting case e.g. [Y M])

				\item the code
			\end{itemize}
	\end{itemize}
\end{frame}

% steam : 
% odd stream: s0
%
% Strem A
% [iso] A * later (Stream A)
% [iso] A * later A * later later (Stream A)
% A * later (Stream A)
% 
% later (later (Stream A)) -> later (Stream A)

% fix f [iso] f (fix f)
% fix[n] f [iso] f (fix[n+1] f)

% prime-el (x, s)
% = if prime number(x) then (x, prime-el s) else prime-el s.
%
% 

% regular expression
%
% syntax
% opeational semantics: AB -> s -> (..)A (...)B -> ok, A+B, A*
% Denotational semantics: r = {s | s ~= r}. AB = {st | s ~= A, t ~= B} A* = fixedpoint
%
% sem[program] = {((c,s), (c',s')) | (c,s) = bigstep => (c',s')}
%
% sem[program] : domain => domain

% stream


\section{reference}
\begin{frame}{reference}
	\begin{itemize}
		\item Marco Paviotti's PhD dissertation
		\item Formalizing \(\pi\)-Calculus in Guarded Cubical Agda
	\end{itemize}
\end{frame}
	
\end{document}
